% Cover letter using letter.sty
\documentclass{letter} % Uses 10pt

\usepackage{url,parskip} 	%other packages for formatting
\RequirePackage{color,graphicx}

% hyperref package for links
\usepackage[pdfborder={0 0 0}]{hyperref}
\definecolor{linkcolour}{rgb}{0,0.2,0.6}
\hypersetup{colorlinks,breaklinks,urlcolor=linkcolour, linkcolor=linkcolour}

%Use \documentstyle[newcent]{letter} for New Century Schoolbook postscript font
% the following commands control the margins:
\topmargin=-1in    % Make letterhead start about 1 inch from top of page 
\textheight=8in  % text height can be bigger for a longer letter
\oddsidemargin=0pt % leftmargin is 1 inch
\textwidth=6.5in   % textwidth of 6.5in leaves 1 inch for right margin

\begin{document}

\signature{Tiago Luchini}                  % name for signature
\longindentation=0pt                       % needed to get closing flush left
\let\raggedleft\raggedright                % needed to get date flush left
 
 
\begin{letter}{Mr. Brandon Carrell \\
Walmart Labs}


\begin{flushleft}
{\large\bf Tiago Luchini}
\end{flushleft}
\medskip\hrule
\begin{flushright}
\hfill 10 City Point, Apt. 27G, Brooklyn, NY 11201 (Green card holder)\\ \hfill
\href{mailto:tiago@luchini.nyc}{tiago@luchini.nyc} ~~
\href{https://luchini.nyc}{https://luchini.nyc} ~~ \href{tel:+16466847728}{+1
  (646) 684 7728}
\end{flushright}
\vfill % forces letterhead to top of page
 
\opening{Dear Brandon,}


\noindent Walmart Labs has been on my radar as an exciting place to
consider working at since I stumbled with Howard Ship’s extraordinary
work on Lacinia. Back then we were considering other stacks for
GraphQL, and Lacinia turned out to be a great asset to us.

\noindent Since then, we’ve built two large scale systems on top of
Lacinia, and I even talked about some of our work on the last two
Clojure/conjs. Here are the links to my talks:

\begin{itemize}
  \item \href{https://www.youtube.com/watch?v=uL9QavmAInw}{The Power
    of Lacinia \& Hystrix in Production}
  \item \href{https://www.youtube.com/watch?v=EDojA_fahvM}{Declarative
    Domain Modeling for Datomic Ion/Cloud}
\end{itemize}

\noindent Despite a long career in leadership roles and using several
stacks and languages, during the last three years I have been coding
exclusively in Clojure with some ClojureScript. My Clojure work has
been around data-intensive applications and tools for other developers
making me very comfortable around custom tooling, DSLs, APIs,
data-pipelines, distributed systems, and resilient applications.

\noindent Some of my Open Source work of notice in this space is:

\begin{itemize}
\item \href{https://github.com/luchiniatwork/hodur-engine}{Hodur}, a
  domain modeling approach and collection of libraries to Clojure. By
  using Hodur you can define your domain model as data, parse and
  validate it, and then either consume your model via an API or use
  one of the many plugins to help you achieve mechanical results
  faster and in a purely functional manner.

\item \href{https://github.com/luchiniatwork/terra}{Terra} lets
  developers and DevOps engineers write Terraform configurations in
  pure Clojure by leveraging the power of Clojure macros. It extends
  the reach of infrastructure-by-code with the power of Clojure.

\item \href{https://github.com/luchiniatwork/migrana}{Migrana} is a
  Datomic migration tool that gives developers control over how their
  Datomic schema evolves. By giving a temporal timeline of schema
  evolutive steps and letting developers tap into those steps, it
  offers a simple alternative to manual schema management.

\item \href{https://github.com/luchiniatwork/cambada}{Cambada}, a
  packager for Clojure based on deps.edn (AKA tools.deps). Supporting
  jar, uberjar, and - particularly - GraalVM’s native-image.

\item
  \href{https://github.com/luchiniatwork/resilience4clj-circuitbreaker}{Resilience4Clj}
  is a lightweight fault tolerance library set built on top of
  GitHub’s Resilience4j, inspired by Netflix Hystrix, and designed for
  Clojure and functional programming with composability in mind.

\end{itemize}

\noindent My current Open Source aspirations are around domain
modeling abstractions, their low-level APIs, means of successfully
exposing them to framework creators and the creation of
highly-resilient applications. More of my open source work can be
found on \href{https://luchini.nyc}{my website}.

\noindent It would be a pleasure to engage in a personal
interview. I’m pretty flexible as to the time and place. Feel free to
reach me by email or phone. Let me know if I can be of assistance to
help in a speedy response. Thank you for taking the time to consider
my credentials.

\closing{Sincerely yours,}


\end{letter}


\end{document}
