% Cover letter using letter.sty
\documentclass{letter} % Uses 10pt

\usepackage{url,parskip} 	%other packages for formatting
\RequirePackage{color,graphicx}

% hyperref package for links
\usepackage[pdfborder={0 0 0}]{hyperref}
\definecolor{linkcolour}{rgb}{0,0.2,0.6}
\hypersetup{colorlinks,breaklinks,urlcolor=linkcolour, linkcolor=linkcolour}

%Use \documentstyle[newcent]{letter} for New Century Schoolbook postscript font
% the following commands control the margins:
\topmargin=-1in    % Make letterhead start about 1 inch from top of page 
\textheight=8in  % text height can be bigger for a longer letter
\oddsidemargin=0pt % leftmargin is 1 inch
\textwidth=6.5in   % textwidth of 6.5in leaves 1 inch for right margin

\begin{document}

\signature{Tiago Luchini}                  % name for signature
\longindentation=0pt                       % needed to get closing flush left
\let\raggedleft\raggedright                % needed to get date flush left
 
 
\begin{letter}{Mr. David Taylor \\
Apple Inc. \\
1 Infinite Loop \\
Cupertino, CA}


\begin{flushleft}
{\large\bf Tiago Luchini}
\end{flushleft}
\medskip\hrule
\begin{flushright}
\hfill 10 City Point, Apt. 31K, Brooklyn, NY 11201 (Green card holder)\\ \hfill
\href{mailto:info@tiagoluchini.eu}{info@tiagoluchini.eu} ~~
\href{https://luchini.nyc}{https://luchini.nyc} ~~ \href{tel:+16466847728}{+1
  (646) 684 7728}
\end{flushright}
\vfill % forces letterhead to top of page
 
\opening{Dear Mr. Taylor:}


\noindent Apple's interest in Clojure is something worth noticing. The best tech
company on the planet investing on the most productive language out there can
only mean there will be an exciting match. In fact, such an exciting match
motivates me to consider a career move despite not being on the market.

\noindent Despite a long career in leadership roles and using several stacks and
languages, during the last two years I have been coding exclusively in Clojure
with some ClojureScript. My Clojure work has been around data-intensive
applications and tooling for other developers making me very comfortable around
macros, DSLs, APIs, data-pipelines, distributed systems and the eventual JVM
optimization.

\noindent In fact, some of my Open Source work is of notice here:

\begin{itemize}
  \item \href{https://github.com/workco/umlaut}{Umlaut}, a neutral schema
    parsing tool that outputs diagrams and code, has been downloaded more than
    4,000 times and is used by developers to model their domain problems.
  \item \href{https://github.com/luchiniatwork/terra}{Terra} lets developers and
    DevOps engineers write Terraform configurations in pure Clojure by
    leveraging the power of Clojure macros. It extends the reach of
    infrastructure-by-code with the power of Clojure.
  \item \href{https://github.com/luchiniatwork/migrana}{Migrana} is a Datomic
    migration tool that gives developers the control over how their Datomic
    schema evolves. By giving a temporal timeline of schema evolutive steps and
    letting developers tap into those steps, it offers a simple alternative to
    manual schema management.
\end{itemize}

\noindent My current Open Source aspiration is around domain modeling
abstractions, their low-level APIs, and means of successfully exposing them to
framework creators. More of my open source work can be found at
\href{https://luchini.nyc}{my website}.

\noindent Last year I was invited to present at the Clojure/conj the work I had
been doing with Hystrix and Lacinia. The
\href{https://github.com/luchiniatwork/conj2017}{video of the talk is here}.

\noindent It would be a pleasure to engage in a personal interview. I'm pretty
flexible as to the time and place. Feel free to reach me by email or phone. Let
me know if I can be of assistance to help in a speedy response. Thank you for
taking time to consider my credentials.
 
\closing{Sincerely yours,}



\end{letter}


\end{document}
